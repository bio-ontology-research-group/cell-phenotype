\documentclass{article}
\begin{document}

\title{Supplement to Semantic integration of physiology phenotypes with an
  application to the Cellular Phenotype Ontology}

\author{Robert Hoehndorf$^{1}$\footnote{to whom correspondence should
    be addressed}, Midori A. Harris$^2$, Heinrich Herre$^3$, \\Gabriella
  Rustici$^4$ and Georgios V. Gkoutos$^{1}$}

\date{$^{1}$Department of Genetics, University of Cambridge,
  Downing Street, Cambridge, Cambridge CB2 3EH, UK\\
  $^{2}$Department of Biochemistry; University of Cambridge, 80 Tennis
  Court Road, Cambridge CB2 1GA, UK\\
  $^{3}$Institute for Medical Informatics, Statistics and
  Epidemiology, University of Leipzig, Haertelstrasse 16-18, 04107
  Leipzig, Germany\\
  $^{4}$European Bioinformatics Institute, Wellcome Trust Genome
  Campus, Hinxton, Cambridge, Cambridge CB10 1SD, UK}

\maketitle

\section{Axiom patterns}

\subsection{Basic pattern}
Our basic pattern for formalizing phenotypes is derived from previous
work \cite{Hoehndorf2010phene}. We use the relation {\bf phenotype-of}
as a relation between a phenotype and some entity (the bearer of the
phenotype). We assume that the bearer of the phenotype is an
organism. Using the {\bf phenotype-of} relation, we further specify
the bearer's properties axiomatically. For example, a phenotype {\em
  Broad forehead} is a phenotype of an entity (the organism) that has
a nose as part which has a quality of being {\em Broad}:
\begin{verbatim}
'Broad forehead' EquivalentTo:
  phenotype-of some (has-part some
    Nose and has-quality some Broad)
\end{verbatim}

\subsection{Comparison to normal}
Phenotypes are often comparative to some reference that is considered
as {\em normal} within a context, while the phenotypes specify
deviations from this normality. One possibility to express deviation
from normality is to explicitly contextualize the classes that
represent the reference entity and either explicitly state or infer
that entities with abnormal phenotypes are distinct from the reference
entities \cite{Hoehndorf2010phene}. Being abnormal is then equivalent
to ``not being normal''. Through the explicit use of the {\bf
  has-part} relation, it becomes possible to formally include {\em
  absence} as a type of abnormality: an {\em Abnormality of X} would
be equivalent to {\em not having a normal X as part}, while the {\em
  Absence of X} is equivalent to {\em not having X as part}. The
resulting inference, that {\em Absence of X} is a subclass of {\em
  Abnormality of X}, is widely implemented in phenotype ontologies
and should be accommodated by our axiom patterns. 

However, the explicit contextualization of classes with {\em normal}
and {\em abnormal} qualifiers requires the use of disjunction, and the
formalization of {\em Abnormality of X} as {\em not having a normal X
  as part} requires the use of negation. Neither operation is
supported in the OWL EL profile \cite{owlprofiles}, and since we limit
ourselves to the OWL EL profile due to its low computational
complexity, we cannot implement these patterns.  As an alternative, we
chose to implement abnormality and absence following formalization
patterns already implemented in phenotype ontologies
\cite{Mungall2010, Hoehndorf2011phenome}, since these patterns can be
expressed in OWL EL. As results, we can perform automated reasoning
over our formalizaton and benefit from interoperability with previous
work on formalizing phenotype ontologies. On the other hand, the use
of these patterns may lead to unintended consequences for some queries
\cite{Boeker2011}.  Furthermore, although we could distinguish between
{\em normal} and {\em abnormal} properties of entities using the PATO
framework, we can also omit this distinction when we build an ontology
that is solely composed of abnormal phenotypes or in which entities
only have abnormal attributes.  Based on these considerations, an
example formalization for {\em Abnormality of cell cycle} is
\begin{verbatim}
'Abnormality of X' EquivalentTo:
  phenotype-of some ( has-part some
    ( part-of some 'Cell cycle' and
      has-quality some Quality))
\end{verbatim}
The use of the class {\tt part-of some 'Cell cycle'} instead of {\tt
  'Cell cycle'} allows us to reuse the {\bf part-of} relation to infer
that abnormalities of parts of the cell cycle become subclasses of
abnormalities of the cell cycle.

\subsection{Definition of complex properties and classes}
We use several relations in order to construct complex classes. In
particular, we use the {\bf regulates} relation from the GO to define
classes of regulation processes, and we use the {\bf towards} to
specify a required second argument in some qualities. 
For example, we define the class {\em Decreased frequency of DNA
  ligation} as:
\begin{verbatim}
'Decreased frequency of DNA ligation' EquivalentTo:
  phenotype-of some (has-part some (participates-in some 
    ((has-quality some 
        ('decreased frequency'
         and (towards some 'DNA ligation')))
     and (regulates some 'DNA ligation'))))
\end{verbatim}
This definition states that a {\em Decreased frequency of DNA
  ligation} is a phenotype of organisms that have a part that
participates in a {\em Regulation of DNA ligation} process which has a
{\em Decreased frequency of DNA ligation} ({\tt 'decreased frequency'
  and (towards some 'DNA ligation')}) as quality. In this definition,
we refer to the class {\em Regulation of DNA ligation} based on its
definition {\tt regulates some 'DNA ligation'}, and to the class {\em
  Decreased frequency of DNA ligation} using the class description
{\tt 'decreased frequency' and towards some 'DNA ligation'}. Although
the use of the {\bf towards} relation in combination with a quality
such as {\em Decreased frequency} has several problems
\cite{obml2011h3}, it will lead to interoperability with the large
number of phenotype ontologies that follow a similar pattern
\cite{Mungall2010}.

\subsection{Chemical substances}
We use the ChEBI ontology \cite{Degtyarenko2007} to refer to classes
that represent chemical entities or substances. In ChEBI, no
distinction is made explicit between chemical entities, in the sense
of individual molecules, and substances composed of these entities.
In some applications, it can be necessary to distinguish between both
in ontology-based information systems or tasks that react differently
to both types (e.g., that distinguish between molecular weights and
melting points) \cite{}. However, we reuse GO axioms that utilize
ChEBI, and the GO does not make a distinction between the molecular
entity and the substance composed of it. Similarly, we do not require
such a distinction to achieve the aims of our
axiomatization. Therefore, we implicitly treat classes in ChEBI as either
molecular entities or substances composed thereof. If this distinction
is made explicit, we can define each class that we currently use from
ChEBI as a disjunction of the class representing a molecular entity
and the substance composed of the entity.

\subsection{Detailed axiom patterns}
Here, we present the precise formulation of axiom patterns. We use $X$
and $A$ as variables in these patterns that are replaced with specific
classes from the GO or ChEBI ontologies.

\paragraph{Phenotype of X:}
If $X$ is a material object, we assert:
\begin{verbatim}
'X phenotype' EquivalentTo:
  phenotype-of some (has-part some
    (part-of some X))
\end{verbatim}

If $X$ is a cellular biological process, we assert:
\begin{verbatim}
'X phenotype' EquivalentTo:
  phenotype-of some (has-part some
    (participates-in some (part-of some X)))
\end{verbatim}

\paragraph{Abnormality of X:}
If $X$ is a material object:
\begin{verbatim}
'X phenotype' EquivalentTo:
  phenotype-of some (has-part some
    (part-of some (X and has-quality some Quality)))
\end{verbatim}

If $X$ is a process, we create a class that does not have any formal
constraints. In CPO, abnormalities of $X$ processes include both
abnormalities of $X$ and abnormality of regulation of $X$. A
formulation of this constraint requires the use of disjunction. Since
we limit ourselves to the use of OWL EL and disjunction is not a valid
logical operator in OWL EL, we do not add any axiom to CPO in this case.

\paragraph{Abnormal morphology and physiology, and absence:}
Abnormal $X$ morphology:
\begin{verbatim}
'X phenotype' EquivalentTo:
  phenotype-of some (has-part some
    (X and has-quality some Morphology))
\end{verbatim}

Abnormal $X$ physiology:
\begin{verbatim}
'X phenotype' EquivalentTo:
  phenotype-of some (has-part some
    (part-of some (X and has-quality some Functionality)))
\end{verbatim}

Formalizing absence of $X$ requires the use of negation. Since
negation is not allowed in OWL EL, we create a class for absence that
we directly assert to be a subclass of {\em Abnormality of X}, without
adding further axioms.

\paragraph{Single occurrence abnormalities:}
Abnormality of single occurrence of $X$:
\begin{verbatim}
'Abnormality of single occurrence of X' EquivalentTo:
phenotype-of some (has-part some (
participates-in some (part-of some X and
has-quality some Quality)))
\end{verbatim}

Increased duration of X:
\begin{verbatim}
'Increased duration of X' EquivalentTo:
phenotype-of some (has-part some (
participates-in some (X and
has-quality some 'Increased duration')))
\end{verbatim}

Decreased duration of X:
\begin{verbatim}
'Decreased duration of X' EquivalentTo:
phenotype-of some (has-part some (
participates-in some (X and
has-quality some 'Decreased duration')))
\end{verbatim}

\paragraph{Abnormalities of X regulation:}
Abnormality of regulation of $X$:
\begin{verbatim}
'Abnormality of regulation of X' EquivalentTo:
phenotype-of some (has-part some (
participates-in some (regulates some X and
has-quality some (Quality and towards some X))))
\end{verbatim}

Increased frequency of $X$:
\begin{verbatim}
'Increased frequency of X' EquivalentTo:
phenotype-of some (has-part some (
participates-in some (regulates some X and
has-quality some ('Increased frequency' and towards some X))))
\end{verbatim}

Decreased frequency of $X$:
\begin{verbatim}
'Decreased frequency of X' EquivalentTo:
phenotype-of some (has-part some (
participates-in some (regulates some X and
has-quality some ('Decreased frequency' and towards some X))))
\end{verbatim}

Late onset of $X$:
\begin{verbatim}
'Late onset of X' EquivalentTo:
phenotype-of some (has-part some (
participates-in some (regulates some X and
has-quality some (Delayed and towards some X))))
\end{verbatim}

Early onset of $X$:
\begin{verbatim}
'Early onset of X' EquivalentTo:
phenotype-of some (has-part some (
participates-in some (regulates some X and
has-quality some (Premature and towards some X))))
\end{verbatim}

\paragraph{Abnormality of inputs and outputs (rates):}
If, by virtue of GO-XP definitions, we are able to identify inputs and
outputs of $X$, we define the following classes:

Increased mass of $A$ as output in regulation of $X$:
\begin{verbatim}
'Decreased mass of A as output in regulation of X' EquivalentTo:
phenotype-of some (has-part some (
participates-in some (regulates some X and
has-output some (A and has-quality some 'Increased mass'))))
\end{verbatim}

Decreased mass of $A$ as output in regulation of $X$:
\begin{verbatim}
'Decreased mass of A as output in regulation of X' EquivalentTo:
phenotype-of some (has-part some (
participates-in some (regulates some X and
has-output some (A and has-quality some 'Decreased mass'))))
\end{verbatim}

Increased mass of $A$ as output in $X$:
\begin{verbatim}
'Increased mass of A as output in X' EquivalentTo:
phenotype-of some (has-part some (
participates-in some (X and
has-output some (A and has-quality some 'Increased mass'))))
\end{verbatim}

Decreased mass of $A$ as output in $X$:
\begin{verbatim}
'Decreased mass of A as output in X' EquivalentTo:
phenotype-of some (has-part some (
participates-in some (X and
has-output some (A and has-quality some 'Decreased mass'))))
\end{verbatim}

We assert similar axioms for the inputs of $X$.

\bibliographystyle{plain}
\bibliography{/home/leechuck/Documents/papers/bibtex/lc}

\end{document}
